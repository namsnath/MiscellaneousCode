\documentclass[ProgramminAssignment.tex]{subfiles}
\begin{document}

\section{Queue (Linked List)}
Write a Program to implement Queue using Linked List (Node class used directly, can be found in Other Resources)

\subsection{Algorithm}
\begin{easylist}
\ListProperties(Start1=1, Start2=1, Start3=1, Start4=1, Start5=1)

	& Display Menu for user to select operation
	& For Insertion, take input of data and call insertAtEnd(), pssing data as parameter
	& For deletion, print data in first Node and call deleteFromStart()
	& For displaying, call display()

\end{easylist}

\subsubsection*{void display():}
\begin{easylist}
\ListProperties(Start1=1, Start2=1, Start3=1, Start4=1, Start5=1)

	& If Size = 0, display appropriate message
	& Else
		&& Create a new Node(pointer) and set it equal to start of the List
		&& Run a loop from i = 0 to size - 1, incrementing i at each iteration
			&&& Print data in pointer
			&&& Set pointer to the next Node in the List
	
\end{easylist}	

\subsubsection*{void deleteFromStart():}
\begin{easylist}
\ListProperties(Start1=1, Start2=1, Start3=1, Start4=1, Start5=1)

	& If List is empty, display appropriate message
	& Else
		&& Set start to the next Node in the List
		&& Decrement size by 1
	
\end{easylist}	

\subsubsection*{void insertAtEnd(int val):}
\begin{easylist}
\ListProperties(Start1=1, Start2=1, Start3=1, Start4=1, Start5=1)

	& Create a new Node with the given data
	& Increment size by 1
	& If List is empty
		&& Set Start and End to new Node
	& Else
		&& Set link of End to new Node
		&& Set end to new Node	
	
\end{easylist}	

\subsection{Code}
\begin{lstlisting}
package linkedList;

import java.util.Scanner;

public class Queue
{

	int size;
	Node start, end;
	
	public void QueueRun()	//For Menu-Driven usage
	{
		int choice = 0;
		int val;
		Scanner s = new Scanner(System.in);
		
		do
		{
			System.out.println("1 - Insertion");
			System.out.println("2 - Deletion");
			System.out.println("3 - Display Queue");
			System.out.println("9 - Exit");
			choice = s.nextInt();
			
			switch(choice)
			{
				case 1:		System.out.println("Enter value");	//Takes input for data
							val = s.nextInt();
							insertAtEnd(val);	//Inserts at end of List
							break;
							
				case 2:		System.out.println(start.getData());	//Prints data of first node and removes it
							deleteFromStart();
							break;
							
				case 3:		display();	//To Display queue
							break;
							
				case 9:		System.exit(0);
				
				default:	System.out.println("\nEnter valid Choice");	
			}
		}while(choice != 9);
		s.close();
	}
	
	public void display()
	{
		if(size == 0)
		{
			System.out.println("Empty Queue");
			return;
		}
		else
		{
			System.out.println("\n\n\n");
			Node pointer = start;
			for(int i = 0; i < size; i++)
			{
				System.out.print(pointer.getData() + "\t");
				pointer = pointer.getLink();
			}
			System.out.println();
		}
	}
	
	public void deleteFromStart()
	{
		if(size == 0)
		{
			System.out.println("Queue Empty");
			return;
		}
		
		start = start.getLink();	//Deletes first node, sets start to next Node in list
		size--;
	}
	
	public void insertAtEnd(int val)	//Function to insert Node at end of List
	{
		Node nptr = new Node(val, null);
		size++;
		
		if(start == null)	//Empty List
		{
			start = nptr;
			end = start;
		}
		else
		{
			end.setLink(nptr);
			end = nptr;	//Sets end to new Node
		}
	}
	
}

\end{lstlisting}
\end{document}