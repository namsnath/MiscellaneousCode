\documentclass[ProgramminAssignment.tex]{subfiles}
\begin{document}

\section{PseudoArithmetic Series}
Write a program to check for a PseudoArithmetic Series and display its sum.\\
A PseudoArithmetic series is one where elements from the opposite ends add up to a common sum.

Input:\\
2, 5, 7, 9, 12

Output:\\
It is a Pseudo Arithmetic Series \\
Common sum is 14\\
Total Sum is 42\\

\subsection{Algorithm}
\begin{easylist}
\ListProperties(Start1=1, Start2=1, Start3=1, Start4=1, Start5=1)

	& Take user input of an array(A[ ]) or use Sample Data (Sample data used here)
	& Declare an int(sum) and store sum of first and last elements of array in it
	& Declare a flag for checking if it is a PseudoArithmetic Series or not
	& Run a loop from i = 0 and j = length - 1 till i $\leq$ j, incrementing i and decrementing j by 1 at each iteration
		&& If A[i] + A[j] $\neq$ sum, raise a flag
	& Declare an int(num) to store half the number of elements in case of even number, (Length + 1) / 2 in case of odd
	& If it is a PseudoArithmetic Series, Display message, common sum and total sum	

\end{easylist}

\subsection{Code}
\begin{lstlisting}
public class PseudoArithmeticSeries
{

	public static void main(String[] args)
	{
		int[] A = {2, 5, 7, 9, 12};	//Sample Data
		
		int sum = A[0] + A[A.length - 1];
		boolean flag = true;
		
		
		for(int i = 0, j = A.length - 1; i <= j; i++, j--)	//Checks oposite elements and sums them
			if((A[i] + A[j]) != sum)	//If sums don't match
				flag = false;
		
		int num = (A.length % 2 == 0) ? A.length / 2 : (A.length + 1) / 2;	//Stores number of elements in half the list for calculating sum
		
		if(flag)
			System.out.println("It is a Pseudo Arithmetic Series \nCommon sum is " + sum + "\nTotal Sum is " + (sum * num));
		else
			System.out.println("It is not a Pseudo Arithmetic Series");
	}

}

\end{lstlisting}
\end{document}