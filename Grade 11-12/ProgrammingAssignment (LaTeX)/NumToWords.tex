\documentclass[ProgramminAssignment.tex]{subfiles}
\begin{document}

\section{Numbers to Words}
Write a prgram to take input of a number between 0 and 1000 and print it out in words.

Input:\\
217

Output:\\
Two Hundred Seveteen

\subsection{Algorithm}
\begin{easylist}
\ListProperties(Start1=1, Start2=1, Start3=1, Start4=1, Start5=1)

	& Declare String array s\_units[ ] and store number names from 1-9 with an empty string at the first index (index 0) to correspond with the indices.
	& Declare String array s\_tens[ ] and store number names for tens digits (twenty, thirty, forty, $\dots$ ninety) with two empty strings in the first two indices (Indices 0 and 1).
	& Declare String array s\_Teens[ ] and store number names for numbers from 10-19 corresponding to indices.
	& Take input of number and check its validity and display appropriate message for invalid input.
	& Parse the String into a StringBuffer and reverse it and append 0 till the StringBuffer is of length 3.
	& Reverse the String again and convert the StringBuffer to a Character array digits[ ].
	& If the number contains digit at the hundred's place and is not followed by 10-19, Print the number name of the digit at hundred's place from array s\_units followed by "Hundred", the digit at Ten's place from array s\_Tens and the units digit from array s\_units.
	& If the number contains digit at the hundred's place and is followed by numbers 10-19, Print the number name of the digits at hundred's place from array s\_units followed by "Hundred", the digits in the following two places from the array s\_Teens.
	& If the number does not contain Hundred's Digit and the number does not belong to 10-19, Print the number at the ten's place from the array s\_Tens followed by the number at unit's place from the array s\_units.
	& If the number does not contain Hundred's digit and lies between 10-19, Print the number name from the array s\_Teens.
	

\end{easylist}

\subsection{Code}
\begin{lstlisting}
import java.util.Scanner;

public class NumToWords
{

	public static void main(String[] args)
	{
		String[] s_units = {"", "One", "Two", "Three", "Four", "Five", "Six", "Seven", "Eight", "Nine"};	//For storing number names of digits in the units place excluding those for 10-19
		String[] s_Tens = {"", "", "Twenty", "Thirty", "Forty", "Fifty", "Sixty", "Seventy", "Eighty", "Ninety"};	//For storing number names of digits in the tens place excluding those for 10-19
		String[] s_Teens = {"Ten", "Eleven", "Twelve", "Thirteen", "Fourteen", "Fifteen", "Sixteen", "Seventeen", "Eighteen", "Nineteen"};	//For storing number names of numbers in the range 10-19
		
		Scanner s = new Scanner(System.in);
		
		System.out.println("Enter Number between 0 and 1000 (Exclusive)");
		int n = s.nextInt();
		if(n <= 0 || n >= 1000)		//Checks validity of input
		{
			System.out.println("Value out of Range");
			System.exit(0);
		}
		
		StringBuffer st = new StringBuffer(Integer.toString(n));	//Converts Integer to a StringBuffer
		st.reverse();	//Reverses the String
		int l = st.length();	//Stores length of the String
		
		if(l < 3)	//If the length is less than maximum allowed, appends zeroes till length is at maximum
			for(int i = 0; i < 3 - l; i++)
				st.append(0);
		
		st.reverse();	//Reverses String again
		
		char digits[] = st.toString().toCharArray();	//Converts String to Character Array
		
		if(Character.getNumericValue(digits[1]) != 1 && Character.getNumericValue(digits[0]) != 0) //Contains Hundreds Digits and does not belong to 10-19
			System.out.println(s_units[Character.getNumericValue(digits[0])] + " Hundred " + s_Tens[Character.getNumericValue(digits[1])] + " " + s_units[Character.getNumericValue(digits[2])]);
		
		if(Character.getNumericValue(digits[1]) == 1 && Character.getNumericValue(digits[0]) != 0) // Contains Hundreds Digits and belongs to 10-19
			System.out.println(s_units[Character.getNumericValue(digits[0])] + " Hundred " + s_Teens[Character.getNumericValue(digits[2])]);
		
		if(Character.getNumericValue(digits[1]) != 1 && Character.getNumericValue(digits[0]) == 0) // Does not Contain Hundred's Digit and does not Belong to 10-19
			System.out.println(s_Tens[Character.getNumericValue(digits[1])] + " " +  s_units[Character.getNumericValue(digits[2])]);
		
		if(Character.getNumericValue(digits[1]) == 1 && Character.getNumericValue(digits[0]) == 0) //Does Not Contain Hundred's Digit and Belongs to 10-19
			System.out.println(s_Teens[Character.getNumericValue(digits[2])]);
	}

}

\end{lstlisting}
\end{document}