\documentclass[ProgramminAssignment.tex]{subfiles}
\begin{document}

\section{Calendar Generator}


Input:\\
Year: 2016\\
Month: January\\
First Day Of Month: Wednesday

Output:\\
\begin{tabular}{ccccccc}
\hline
\multicolumn{7}{c}{January 2016}\\ \hline
Sun & Mon & Tue & Wed & Thu & Fri & Sat \\
    &     &     & 1   & 2   & 3   & 4   \\
5   & 6   & 7   & 8   & 9   & 10  & 11  \\
12  & 13  & 14  & 15  & 16  & 17  & 18  \\
19  & 20  & 21  & 22  & 23  & 24  & 25  \\
26  & 27  & 28  & 29  & 30  & 31  &    
\end{tabular}

\subsection{Algorithm}
\begin{easylist}
\ListProperties(Start1=1, Start2=1, Start3=1, Start4=1, Start5=1)

	& Declare a Scanner object to take user input
	& Take user inputs for Year, Month and First day of the month
	& Check if the year is a leap year and assign a value to boolean(isLeapYear)
	& Declare a 5x7 int array(cal[ ][ ]) to write the calendar to
	& Declare two ints (days and startPos) to store number of days in the month and starting position in the calendar respectively
	& Assign days its value based on Month and if it is a leap year or not
	& Assign startPos according to the day of the week (Starting from Sunday = 0)
	& Run a loop from i = 0 and count = 1, till count is $\leq$ days, incrementing i at each iteration
		&& if i = 0 (First row)
			&&& Run a loop from j = startPos, till j $<$ 7 and count $\leq$ days, incrementing j and count at each iteration
				&&&& set cal[i][j] to count
		&& Else
			&&& Run a loop from j = 0, till j $<$ 7 and count $\leq$ days, incrementing j and count at each iteration
				&&&& set cal[i][j] to count
	& Print the array in given format					

\end{easylist}

\subsection{Code}
\begin{lstlisting}
import java.util.Scanner;

public class CalendarGenerator
{

	public static void main(String[] args)
	{
		Scanner s = new Scanner(System.in);
		
		System.out.println("Enter Year");
		int year = s.nextInt();					//To store year
		System.out.println("Enter Month");
		String month = s.next().trim();			//To store month
		System.out.println("Enter First Day of Month");
		String day = s.next().trim();			//To store first day of the month
		
		boolean isLeapYear = (year % 400 == 0) || ((year % 4 == 0) && (year % 100 != 0));	//To check for leap year
		
		int cal[][] = new int[5][7];			//Creates an Array to store the dates
		
		int days, startPos = 0;					//Days to store number of days in month, startPos to store start position of dates in array
		
		if(month.equalsIgnoreCase("January") || month.equalsIgnoreCase("March") || month.equalsIgnoreCase("May") || month.equalsIgnoreCase("July") || month.equalsIgnoreCase("August")
				|| month.equalsIgnoreCase("October") || month.equalsIgnoreCase("December"))		//Checks for months with 31 days
			days = 31;
		else if(month.equalsIgnoreCase("February") && isLeapYear)		//Checks for February in Leap Year
			days = 29;
		else if(month.equalsIgnoreCase("February") && !isLeapYear)		//Checks for February in non-Leap year
			days = 28;
		else
			days = 30;		//else month has 30 days
		
		switch(day.toLowerCase())		//To assign startPos based on Day
		{
			case "sunday":	startPos = 0; break;
			case "monday":	startPos = 1; break;
			case "tuesday":	startPos = 2; break;
			case "wednesday":	startPos = 3; break;
			case "thursday":	startPos = 4; break;
			case "friday":	startPos = 5; break;
			case "saturday":	startPos = 6; break;
		}
		
		for(int i = 0, count = 1; count <= days; i++)	//Loop to fill dates in array
		{
			if(i == 0)
				for(int j = startPos; j < 7 && count <= days; j++, count++)		//For 1st row
					cal[i][j] = count;
			else
				for(int j = 0; j < 7 && count <= days; j++, count++)		//For other rows
					cal[i][j] = count;
		}
		
		
		//Printing
		System.out.println("-------------------------------------------------------");
		System.out.println("\t\t " + month + " " + year + "");
		System.out.println("-------------------------------------------------------");
		System.out.println("Sun\tMon\tTue\tWed\tThu\tFri\tSat");
		
		for(int i = 0; i < 5; i++)
		{
			for(int j = 0; j < 7; j++)
			{
				if(cal[i][j] == 0)
					System.out.print("\t");
				else
					System.out.print(cal[i][j] + "\t");
			}
			System.out.println();
				
		}
		
		s.close();
	}

}

\end{lstlisting}
\end{document}