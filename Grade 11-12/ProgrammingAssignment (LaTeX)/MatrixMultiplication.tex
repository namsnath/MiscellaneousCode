\documentclass[ProgramminAssignment.tex]{subfiles}
\begin{document}

\section{Matrix Multiplication}
Write a Program to multiply two matrices.

Input:\\
Matrix 1:\\
\begin{tabular}{ccc}
1&	3&	5\\	
6&	7&	8\\	
9&	4&	5	\\
\end{tabular}

Matrix 2:\\
\begin{tabular}{ccc}
2&	12&	6\\	
8&	9&	7	\\
3&	4&	5	\\
\end{tabular}

Output:\\
Multiplied Matrix:\\
\begin{tabular}{ccc}
41&	59&	52\\	
92&	167&	125\\	
65&	164&	107	\\
\end{tabular}

\subsection{Algorithm}
\begin{easylist}
\ListProperties(Start1=1, Start2=1, Start3=1, Start4=1, Start5=1)

	& Declare a Scanner object to take user input
	& Take user input for dimensions of the two matrices and check if columns in first = rows in second
	& Create two arrays (A[ ][ ] and B[ ][ ]) of appropriate size
	& Take user input for the elements of the matrices
	& Print the original matrices
	& Create an array (C[ ][ ]) for the multiplication of the size Rows1 $\times$ Columns2
	& Run a loop from i = 0 till Rows1 - 1, incrementing i at each iteration
		&& Run a loop from j = 0 till Columns2 - 1, incrementing j at each iteration
			&&& Run a loop from k = 0 till  Rows1 - 1, incrementing k at each iteration
				&&&& Add A[i][k] $\times$ B[k][j] to C[i][j] (Row of first matrix with column of second)
	& Print the Multiplied Matrix			

\end{easylist}

\subsection{Code}
\begin{lstlisting}
import java.util.Scanner;

public class Multiplication
{

	public static void main(String[] args)
	{
		Scanner s  = new Scanner(System.in);
		int r1, r2, c1, c2, i, j;
		
		//Dimensions of the two Matrices
		System.out.println("Enter Number of Rows in First Matrix:");
		r1 = s.nextInt();
		
		System.out.println("Enter Number of Columns in First Matrix:");
		c1 = s.nextInt();
		
		System.out.println("Enter Number of Rows in Second Matrix:");
		r2 = s.nextInt();
		
		System.out.println("Enter Number of Columns in Second Matrix:");
		c2 = s.nextInt();
		
		
		if(c1 != r2)	//Condition for multiplication
		{
			System.out.println("Number of Columns in Matrix 1 have to be equal to Rows in Matrix 2");
			System.exit(0);
		}
		
		//Creating the Matrices
		int A[][] = new int[r1][c1];
		int B[][] = new int [r2][c2];
		
		//Data Entry
		System.out.println("Enter the elements in the First Matrix, Row-Wise:");
		
		for(i = 0; i < r1; i++)
			for(j = 0; j < c1; j++)
				A[i][j] = s.nextInt();
		
		
		System.out.println("Enter the elements in the Second Matrix, Row-Wise:");
		
		for(i = 0; i < r2; i++)
			for(j = 0; j < c2; j++)
				B[i][j] = s.nextInt();
		
		System.out.println("\n\n\n");
		
		//Printing the Original Matrices
		System.out.println("Matrix 1:");
		
		for(i = 0; i < r1; i++)
		{
			for(j = 0; j < c1; j++)
				System.out.print(A[i][j] + "\t");
			System.out.println();
		}
		
		System.out.println("Matrix 2:");
		
		for(i = 0; i < r2; i++)
		{
			for(j = 0; j < c2; j++)
				System.out.print(B[i][j] + "\t");
			System.out.println();
		}
		
		int C[][] = new int [r1][c2];	//Multiplication array
		
		for( i = 0; i < r1; i++)
		{
			for( j = 0;j < c2; j++)
			{
				for(int k = 0;k < r1; k++)
				{
		            C[i][j] += A[i][k] * B[k][j];	//Multiplies Row of 1st matrix with Column of the other and adds to third
		        }
		    }
		}
		
		System.out.println("Multiplied Matrix:");	//Printing the Multiplied Matrix
		
		for(i = 0; i < r1; i++)
		{
			for(j = 0; j < c2; j++)
				System.out.print(C[i][j] + "\t");
			System.out.println();
		}
		
		s.close();
	}

}

\end{lstlisting}
\end{document}