\documentclass[ProgramminAssignment.tex]{subfiles}
\begin{document}

\section{Special Numbers}
Write a program to check if a given number is a Special Number.\\
A number is a Special number if the sum of the Factorials of it's digits equals the number itself.

Input:\\
145

Output:\\
145 is a Special Number\\
1! + 4! + 5! = 1 + 24 + 120 = 145

\subsection{Algorithm}
\begin{easylist}
\ListProperties(Start1=1, Start2=1, Start3=1, Start4=1, Start5=1)

	& Take user input for a number
	& Declare an int(sum) to store the sums of the factorials of the numbers
	& Extract each digits from the number and find their factorials, adding each to sum
	& Check if the sum equals the number

\end{easylist}

\subsubsection*{int factorial(int n):}
\begin{easylist}
\ListProperties(Start1=1, Start2=1, Start3=1, Start4=1, Start5=1)

	& If n $<$ 2, return 1
	& Else return n * factorial(n - 1)
 	
\end{easylist}	

\subsection{Code}
\begin{lstlisting}
import java.util.Scanner;
public class SpecialNum
{

	public static void main(String[] args)
	{
		Scanner s = new Scanner(System.in);
		
		System.out.println("Enter Number to check:");
		int n = s.nextInt();	//Data entry
		
		int sum = 0;	//To store sum of factorials
		
		String st = String.valueOf(n);	//Converts to String
		for(int i = 0; i < st.length(); i++)
		{
			sum += factorial(Integer.parseInt(st.charAt(i) + ""));	//Extracts each digit and adds its factorial
		}
		
		if(n == sum)
			System.out.println(n + " is a Special Number");
		else
			System.out.println(n + " is not a Special Number");
	}
	
	static int factorial(int n)	//Recursive function to calculate factorial
	{
		if(n < 2)
			return 1;
		
		return n * factorial(n - 1);
	}	
}
\end{lstlisting}
\end{document}