\documentclass[ProgramminAssignment.tex]{subfiles}
\begin{document}

\section{Merge Sort (Array)}
Write a Program to sort an array using MergeSort




\subsection{Algorithm}
\begin{easylist}
\ListProperties(Start1=1, Start2=1, Start3=1, Start4=1, Start5=1)

	& Declare an int array A and take user input
	& Call the MergeSort function, passing A, 0, A.length - 1 as argument
	& Print the Array

\end{easylist}

\subsubsection*{MergeSort(int[ ] A, int p, int r):}
\begin{easylist}
\ListProperties(Start1=1, Start2=1, Start3=1, Start4=1, Start5=1)

	& Declare a int variable to store the middle element (int q)
	& If p is more than r, 
		&& Set q to average of p and r (middle element)
		&& Call merge sort on first half of array (p, q)
		&& Call merge sort on second half of array (q + 1, r)
		&& Call merge on entire array (p, q, r)
	
\end{easylist}	

\subsubsection*{Merge(int[ ] A, int p, int q, int r):}
\begin{easylist}
\ListProperties(Start1=1, Start2=1, Start3=1, Start4=1, Start5=1)

	& Declare an int (n1) to store size of first half of array passed (p - q + 1)
	& Declare an int (n2) to store size of second half of array passed (r - q)
	& Declare an int array L of size n1 + 1, and array R of size n2 + 1
	& Fill first half of original array in L, and second half in R
	& Set final element in L and R to Integer.MAX\_VALUE for comparison ease
	& Declare int variable (i and j) and initialize them to 0
	& Run a loop from k = p, to k $\leq$ r, incrementing k at each iteration
	& If L[i] $\leq$ R[j] (Comparing elements in the two arrays)
		&& Set A[k] to L[i] (Smaller element)
		&& Increment i by 1
	& Else
		&& Set A[k] to R[j]
		&& Increment j by 1
	
\end{easylist}	



\subsection{Code}
\begin{lstlisting}
import java.util.Scanner;

public class MergeSortArray
{

	public static void main(String[] args)
	{
		Scanner s = new Scanner(System.in);
		
		System.out.println("Enter number of elements");
		int n = s.nextInt();
		
		int[] A = new int[n];
		
		System.out.println("Enter elements");
		for(int i = 0; i < n; i++)
			A[i] = s.nextInt();
		
		MergeSortArray ms = new MergeSortArray();
		ms.MergeSort(A, 0, A.length - 1);				//Passes the entire array to the function
		
		for(int i = 0; i < A.length; i++)
			System.out.print(A[i]  + " ");
	}

	public void MergeSort(int[] A, int p, int r)
	{
		int q;
		if(p < r)									//Runs till lower index is less then upper index
		{
			q = (p + r) / 2;						//Gets middle index
			MergeSort(A, p, q);						//Passes first half of array to itself
			MergeSort(A, q + 1, r);					//Passes seconds half of aray to itself
			Merge(A, p, q, r);						//Calls merge on entire array with q as middle element
		}
	}
	
	public void Merge(int[] A, int p, int q, int r)
	{
		int n1 = q - p + 1;		//Gets size of first half of array passed
		int n2 = r - q ;		//Gets size of second half of array passed
		int i, j, k;
		
		
		int[] L = new int[n1 + 1], R = new int[n2 + 1];		//Declaring arrays corresponding to halves
		
		//Filling elements in the new arrays
		
		for(i = 0; i < n1; i++)
			L[i] = A[p + i];
		for(j = 0; j < n2; j++)
			R[j] = A[q + j + 1];
		
		
		//Sets last element in array to Maximum value of integer for comparison
		L[n1] = Integer.MAX_VALUE;		
		R[n2] = Integer.MAX_VALUE;
		
		i = 0;
		j = 0;
		
		for(k = p; k <= r; k++)		//Iterates thtough elements in the given range
		{
			//Rewrites elements to original array after comparison of two halves
			
			if(L[i] <= R[j])
			{
				A[k] = L[i];
				i++;
			}
			else
			{
				A[k] = R[j];
				j++;
			}
		}	
				
	}
}

\end{lstlisting}
\end{document}