\documentclass[ProgramminAssignment.tex]{subfiles}
\begin{document}

\section{Sieve Of Erasthosenes}
Write a program to implement the Sieve of Erasthosenes to find all the prime numbers below a given limit and print them out, 
along with the number of primes.

Input:\\
Limit = 25

Output:\\
2  3  5  7  11  13  17  19  23  \\
9 primes

\subsection{Algorithm}
\begin{easylist}
\ListProperties(Start1=1, Start2=1, Start3=1, Start4=1, Start5=1)

	& Declare two int variables (maxNum and count) to store the Maximum limit to display primes up to and the count of primes in the given range, and take user input for maxNum.
	& Check for validity of Input, and display appropriate message for invalid input.
	& Declare a boolean array (primes[ ]) to implement the sieve.
	& Set all values in primes[ ] to true.
	& Set first element in primes (primes[0]) corresponding to the number '1' to false.
	& Run a loop to iterate through primes[ ] and do the following:
		&& Check if current element is true and is a prime number using isPrime() function, go to a) if true
			&&& run a loop from (i+1) * 2 to maxNum, increment by (i+1) on each iteration and set each (j-1) element to false
	& Run a loop to iterate through primes[ ] and print (index + 1) as a numeric for all elements that are true and increment count with each print.
	& Print count to display total number of primes in the limit.

\end{easylist}

\subsubsection*{boolean isPrime(int n):}
\begin{easylist}
\ListProperties(Start1=1, Start2=1, Start3=1, Start4=1, Start5=1)

	& if n is 0 or 1, return false
	& if n is 2, return true
	& if n is a multiple of 2, return false
	& Run a loop from 3 to sqrt(n) and increment by 2 at each iteration
		&& If n is divisible by the current index, return false, else continue loop
	& return true
	
	
\end{easylist}	

\subsection{Code}
\begin{lstlisting}
import java.util.Scanner;

public class SieveOfErasthosenes
{

	public static void main(String[] args)
	{
		Scanner s = new Scanner(System.in);		//Creates Object of Scanner
		
		System.out.println("Enter limit:");
		int maxNum = s.nextInt(), count = 0;
		
		if(maxNum <= 0)
		{
			System.out.println("Entered range is invalid");
			System.exit(0);
		}
		
		boolean[] primes = new boolean[maxNum];			//Creates boolean array to implement the sieve
		for(int i = 0; i < maxNum; i++)					//Sets all values in array to true
			primes[i] = true;
		
		primes[0] = false;							//Sets 1 to false
		
		for(int i = 1; i < maxNum; i++)					//Loop to iterate through array
			if(primes[i] && isPrime(i + 1))			//Checks if current element is true and is a prime
				for(int j = 2 * (i + 1); j <= maxNum; j += (i + 1))	//Sets all multiples of the current prime element to false
					primes[j - 1] = false;
		
		for(int i = 0; i < maxNum; i++)		//Loop to iterate through the array
			if(primes[i])		//Prints the elemnt if it is prime
			{
				System.out.print((i + 1) + "  ");
				count++;
			}
			
		System.out.println("\n" + count + " primes");		//Prints count of primes
	}
	
	public static boolean isPrime(int n)		//Function to check if a given number is Prime
	{
		if(n == 0 || n == 1)
			return false;
		if(n == 2)
			return true;
		if(n % 2 == 0)
			return false;
		for(int i = 3; i * i <= n; i += 2)
			if(n % i == 0)
				return false;
		return true;
	}

}

\end{lstlisting}
\end{document}