\documentclass[ProgramminAssignment.tex]{subfiles}
\begin{document}

\section{Primes Array}
Write a program to accept dimensions of an array and fill it with prime numbers.

Input:\\
Rows = 3\\
Columns = 3

Output:\\
\begin{tabular}{ c c c }
  2 & 3 & 5 \\
  7 & 11 & 13 \\
  17 & 19 & 23 \\
\end{tabular}


\subsection{Algorithm}
\begin{easylist}
\ListProperties(Start1=1, Start2=1, Start3=1, Start4=1, Start5=1)

	& Take input for number of rows and columns from the user and store in two variables (m, n)
	& Create a new 2-D int array (A) of size m * n
	& Create a new int array (primes) of size (m*n)
	& Declare an int (k) and initialize it to 2
	& Run a loop from i = 0 to m*n and increment k at each iteration
		&& Check if current value of k is a prime, if true, go to i)
			&&& Set primes[i] to k
			&&& Increment i by 1
	& Set k to 0
	& Run a loop from i = 0 to m, incrementing i at each iteration
		&& Run a loop from j = 0 to n, incrementing j at each iteration
			&&& Set A[i][j] to primes[k]
			&&& Increment k by 1
	& Print array A 

\end{easylist}

\subsubsection*{boolean checkPrime(int n):}
\begin{easylist}
\ListProperties(Start1=1, Start2=1, Start3=1, Start4=1, Start5=1)

	& if n is 0 or 1, return false
	& if n is 2, return true
	& if n is a multiple of 2, return false
	& Run a loop from 3 to sqrt(n) and increment by 2 at each iteration
		&& If n is divisible by the current index, return false, else continue loop
	& return true
	
\end{easylist}	

\subsection{Code}
\begin{lstlisting}
import java.util.Scanner;
public class PrimesInArray
{
	public static void main(String[] args)
	{
		Scanner s = new Scanner(System.in);
		
		System.out.println("Enter Number of Rows");			//Take input for number of rows
		int m = s.nextInt();
		
		
		System.out.println("Enter Number of Columns");		//Take input for number of columns
		int n = s.nextInt();
		
		int[][] A = new int[m][n];							//Declare array of size of input values
		
		int i, j, k = 2;
		int[] primes = new int[m * n];						//Declare array of size m * n
		
		for(i = 0; i < m * n; k++)							//Run through numbers more than 2
		{
			if(checkPrime(k))								//Check if current number is prime
			{
				primes[i] = k;
				i++;										//Increment counter if number is prime
			}
		}
		
		k = 0;
		
		for(i = 0; i < m; i++)								//Write primes to 2-D array
			for(j = 0; j < n; j++)
				{
					A[i][j] = primes[k];
					k++;
				}
		
		System.out.println("Your Prime Array");
		
		//Print Array
		
		for(i = 0; i < m; i++)
		{
			for(j = 0; j < n; j++)
				System.out.print(A[i][j] + "\t");
			
			System.out.println();
		}
		
	}
	
	public static boolean checkPrime(int n)		//Function to check Primes
	{
		if(n == 2)
			return true;
		if (n % 2 == 0) 
			return false;
	    for(int i = 3; i * i <= n; i += 2) 
	        if(n % i == 0)
	            return false;
	    
	    return true;
	}
}
\end{lstlisting}
\end{document}