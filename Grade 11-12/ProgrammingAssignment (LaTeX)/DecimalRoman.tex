\documentclass[ProgramminAssignment.tex]{subfiles}
\begin{document}

\section{Decimal To Roman Numerals}
Write a Program to Convert decimal numbers to their Roman equivalent

Input:\\
969

Output:\\
CMLXIX

\subsection{Algorithm}
\begin{easylist}
\ListProperties(Start1=1, Start2=1, Start3=1, Start4=1, Start5=1)

	& Declare a String array(Hundreds[ ]) and fill it wih Hundreds in Roman Numerals(C, CC $\dots$ CM)
	& Declare a String array(Tens[ ]) and fill it with Tens in Roman Numerals(X, XX $\dots$ XC)
	& Declare a String array(Units[ ]) and fill it with Units in Roman Numerals(I, II $\dots$ IX)
	& Take user input of number
	& Hundreds is found by taking the element at n / 100 in Hundreds[ ]
	& Tens is found by taking element at (n / 10) \% 10 in Tens[ ]
	& Units is found by taking element at n \% 10 in Units[ ]

\end{easylist}

\subsection{Code}
\begin{lstlisting}
import java.util.Scanner;
public class DecimalRoman
{

	public static void main(String[] args)
	{
		Scanner s = new Scanner(System.in);
		
		String[] Hundreds = {"", "C", "CC", "CCC", "CD", "D", "DC", "DCC", "DCCC", "CM"};	//Array for Hundreds in Roman
		String Tens[]={"", "X", "XX", "XXX", "XL", "L", "LX", "LXX", "LXXX", "XC"};	//Array for Tens in Roman
		String Units[]={"", "I", "II", "III", "IV", "V", "VI", "VII", "VIII", "IX"};	//Array for Units in Roman
		
		System.out.println("Enter Decimal Number to Convert to Roman (Less than 1000):");	//Input
		int n = s.nextInt();
		
		String Hund = Hundreds[n / 100];	//Division by 100 yields number of hundreds
		String Ten = Tens[(n / 10) % 10];	//Division by 10 and the remainder of further division by 10 yields Tens
		String Unit = Units[n % 10];	//Remainder of division by 10 yields number of Ones
		
		System.out.println("Roman Equivalent: " + Hund + Ten + Unit);	//Final Answer
	}

}

\end{lstlisting}
\end{document}