\documentclass[ProgramminAssignment.tex]{subfiles}
\begin{document}

\section{Binary Search (Recursive)}
Write a program to implement recursive binary search in an array.

\subsection{Algorithm}
\begin{easylist}
\ListProperties(Start1=1, Start2=1, Start3=1, Start4=1, Start5=1)

	& Take user input of an array or use a sorted sample array (sample used here)
	& Take user input for data to search for
	& Call binarySearch(), passing the array, element, 0 and length - 1 as parameters
	& Print position if function does not return -1, else display appropriate message

\end{easylist}
\subsubsection*{int binarySearch(int[] A, intx, int low, int high)}
\begin{easylist}
\ListProperties(Start1=1, Start2=1, Start3=1, Start4=1, Start5=1)

	& If low $>$ high (out of bounds), return -1 (not found)
	& Set mid to (low + high) / 2
	& If A[mid] $<$ x, call binarySearch, passing A, x, mid + 1 and high as parameters and return its value
	& If A[mid] $>$ x, call binarySearch, passing A, x, low and mid - 1 as parameters and return its value
	
\end{easylist}	

\subsection{Code}
\begin{lstlisting}
import java.util.Scanner;
public class RecursiveBinarySearch
{
	public static void main(String[] args)
	{
		int A[] = {2, 45, 69, 234, 567, 876, 900, 976, 999};	//Sample Data
		Scanner s = new Scanner(System.in);
		
		System.out.println("Enter data to search for:");
		int x = s.nextInt();
		
		int found = binarySearch(A, x, 0, A.length - 1);
		
		if(found != -1)
			System.out.println("Found at position " + (found + 1));
		else
			System.out.println("Not Found");
	}	

	static int binarySearch(int A[], int x, int low, int high) 
	{
		if(low > high)
			return -1;
		int mid = (low + high) / 2;
		if(x > A[mid])
			return binarySearch(A, x, mid + 1, high);
		else if(x < A[mid])
			return binarySearch(A, x, low, mid - 1);
		return mid;
	}	
}
\end{lstlisting}
\end{document}