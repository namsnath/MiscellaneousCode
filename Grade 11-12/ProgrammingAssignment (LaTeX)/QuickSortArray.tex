\documentclass[ProgramminAssignment.tex]{subfiles}
\begin{document}

\section{Quick Sort(Array)}
Write a program to implement quicksorting in an array.

Input:\\
2	6	4	8	1

Output:\\
1	2	4	6	8

\subsection{Algorithm}
\begin{easylist}
\ListProperties(Start1=1, Start2=1, Start3=1, Start4=1, Start5=1)

	& Declare an int array (A) and take user input
	& Call quicksort on entire array (0, A.length - 1)
	& Print array A
	
\end{easylist}	

\subsubsection*{void quicksort(int[ ] A, int left, int right):}
\begin{easylist}
\ListProperties(Start1=1, Start2=1, Start3=1, Start4=1, Start5=1)

	& Declare an int q
	& If right is more than left,
		&& Set q as Partition(A, left , right) [Partition of entire array]
		&& Call Quicksort on first half (left, q - 1)
		&& Call quicksort on second half (q + 1, right)
	
\end{easylist}		

\subsubsection*{int partition(int[ ] A, int left, int right):}
\begin{easylist}
\ListProperties(Start1=1, Start2=1, Start3=1, Start4=1, Start5=1)

	& Declare an int P (Pivot element) and set it to the first element in the given array (A[left])
	& Delare an int i and set it to left, another int j and set it to right + 1
	& Run an infinite loop
		&& While A[++i] is less than Pivot element,
			&&& Check if i $\geq$ right, if yes, break
		&& While A[$--$j] is more than Pivot element
			&&& Check if j $\leq$ left, if yes, break
		&& If i is $\geq$ j, 
			&&& break
		&& Else, 
			&&& swap I and j
	& if j = left,
		&& return j
	& swap elements at left and j 
	& return j 
	
\end{easylist}	
	

\subsubsection*{void swap(int[ ] A, inti, int j):}
	Swap A[i] and A[j]


\subsection{Code}
\begin{lstlisting}
import java.util.Scanner;
public class QuickSortArray
{

	public static void main(String[] args)
	{
		Scanner s = new Scanner(System.in);
		
		System.out.println("Enter size of array");
		int n = s.nextInt();
		
		int A[] = new int[n];
		
		System.out.println("Enter elements");
		for(int i = 0; i < n; i++)
			A[i] = s.nextInt();
		
		QuickSortArray qs = new QuickSortArray();
		
		qs.quicksort(A, 0, A.length - 1);			//Calls quicksort on given array
		
		for(int i = 0; i < A.length; i++)
			System.out.print(A[i]  + " ");
	}

	public void quicksort(int A[], int left, int right)
	{
		int q;
		if(right > left)					//if upper index is more than lower index
		{
			q = partition(A, left, right);	//Calls partition on array between left and right indices
			quicksort(A, left, q - 1);		//Calls itself on array between left and index of j passed back from partition function
			quicksort(A, q + 1, right);		//Calls itself on array between q + 1 and right
		}
	}
	
	public int partition(int A[], int left, int right)	//Function to split array into 3 parts, left block, pivot, right block
	{
		int P = A[left];		//Set Pivot P to the first element in the given array
		int i = left;			//first index in array
		int j = right + 1;		//last index + 1 in array
		
		for(;;)
		{
			while(A[++i] < P)	//finds last element in line which is smaller than Pivot
				if(i >= right)	//Breaks if index exceeds array range
					break;
			
			while(A[--j] > P)	//Finds last element in line (from the rear) which is larger than Pivot
				if(j <= left)	//Breaks if index is smaller than lowest index of array
					break;
			
			
			if(i >= j)			//Breaks if i and j are same, or i is larger
				break;
			else
				swap(A, i, j);	//Swap elements at i and j to put elements smaller than pivot in 1st subblock and larger than pivot in second subblock
		}
		
		if(j == left)			//If j has gone down to lowest index, returns lowest index
			return j;
		
		swap(A, left, j);		//Swap lowest index and current index of j
		return j;				//Returns j
	}
	
	public void swap(int[] A, int i, int j)		//Function to swap elements at indices i and j
	{
		int temp = A[i];
		A[i] = A[j];
		A[j] = temp;
	}
	
}

\end{lstlisting}
\end{document}