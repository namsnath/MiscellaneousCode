\documentclass[ProgrammingAssignment.tex]{subfiles}

\begin{document}

\section{Decimal, Binary, Octal, Hexadecimal Convertor}

%\subsection*{Question}
Write a program to convert Decimal to Binary, Octal and Hexadecimal numbering systems.\\
Example:

Input:\\
13

Output:\\\
Number in Binary: 1101\\
Number in Octal: 0o15\\
Number in Hexadecimal: 0xD
\subsection{Algorithm}

\begin{easylist}
\ListProperties(Start1=1, Start2=1, Start3=1, Start4=1, Start5=1)
  & Declare a static int array rems[ ] (from �1� to �9� and �A� to �F�) to store remainders for use in conversion
  & Take user input for number to convert
  &	Call all three functions and print the returned values
\end{easylist}  

\subsubsection*{String DecToBin(int n):}
\begin{easylist}
\ListProperties(Start1=1, Start2=1, Start3=1, Start4=1, Start5=1)
  & If n = 0, return �0�
  & Declare a String (Binary), and initialize it to an empty String
  & Run a loop from rem = n \% 2, till n $>$ 0, setting rem to n \% 2 at each iteration
    && Append rems[rem] to the front of Binary
    && Set n = n / 2 
  & Return Binary 
\end{easylist}


\subsubsection*{String DecToOct(int n):}
\begin{easylist}
\ListProperties(Start1=1, Start2=1, Start3=1, Start4=1, Start5=1)
  & If n = 0, return �0�
  & Declare a String (Octal), and initialize it to an empty String
  & Run a loop from rem = n \% 8, till n $>$ 0, setting rem to n \% 8 at each iteration
    && Append rems[rem] to the front of Octal
    && Set n = n / 8 
  & Return ``0o" + Octal 
\end{easylist}

\subsubsection*{String DecToHex(int n):}
\begin{easylist}
\ListProperties(Start1=1, Start2=1, Start3=1, Start4=1, Start5=1)
  & If n = 0, return �0�
  & Declare a String (Hex), and initialize it to an empty String
  & Run a loop from rem = n \% 16, till n $>$ 0, setting rem to n \% 16 at each iteration
    && Append rems[rem] to the front of Hex
    && Set n = n / 16 
  & Return ``0x'' + Hex 
\end{easylist}


\subsection{Code}
\begin{lstlisting}
import java.util.Scanner;

public class DecToBin_Oct_Hex
{
	static char[] rems = {'0', '1', '2', '3', '4', '5', '6', '7', '8', '9', 'A', 'B', 'C', 'D', 'E', 'F'};	//Array
	
	public static void main(String args[])
	{
		Scanner s = new Scanner(System.in);
		
		System.out.println("Enter Number to Convert");
		int n = s.nextInt();
		
		System.out.println("Number in Binary: " + Q11_DecToBin_Oct_Hex.DecToBin(n));	//Prints Binary equivalent by calling Function
		System.out.println("Number in Octal: " + Q11_DecToBin_Oct_Hex.DecToOct(n));		//Prints Octal equivalent by calling Function
		System.out.println("Number in Hexadecimal: " + Q11_DecToBin_Oct_Hex.DecToHex(n));	//Prints Hexadecimal equivalent by calling Function
		
		s.close();
	}
	
	public static String DecToBin(int n)	//Function to return Binary equivalent of a decimal number
	{
		if(n == 0)
			return "0";
		
		String Binary = "";	//String to store Binary equivalent
		for(int rem = n % 2; n > 0; rem = n % 2)	//Loop to append remainder to string, creating Binary number
		{
			Binary = rems[rem] + Binary; 
			n = n / 2; 
		}
		return Binary;	//returns Binary equivalent
	}
	
	public static String DecToOct(int n)	//Function to return Octal equivalent of a decimal number
	{
		if(n == 0)
			return "0";
		
		String Octal = "";	//String to store Octal equivalent
		for(int rem = n % 8; n > 0; rem = n % 8)	//Loop to append remainder to string, creating Octal number
		{
			Octal = rems[rem] + Octal;
			n = n / 8; 
		}
		return "0o" + Octal;	//returns Octal equivalent
	}
	
	public static String DecToHex(int n)	//Function to return Hexadecimal equivalent of a decimal number
	{
		if(n == 0)
			return "0";
		
		String Hex = "";	//String to store Hexadecimal equivalent
		for(int rem = n % 16; n > 0; rem = n % 16)	//Loop to append remainder to string, creating Hexadecimal number
		{
			Hex = rems[rem] + Hex;
			n = n / 16;
		}
		return "0x" + Hex;	//returns Hexadecimal equivalent
	}
}

\end{lstlisting}

\end{document}