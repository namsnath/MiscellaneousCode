\documentclass[ProgrammingAssignment.tex]{subfiles}

\begin{document}

\section{Determinant Solver}
Write a program to solve a determinant of order 'n' recursively

Input:\\
\begin{tabular}{c c c c}
1 & 2 & 3 & 4\\
2 & 3 & 4 & 5\\ 
5 & 7 & 2 & 9\\ 
4 & 8 & 3 & 6
\end{tabular}

Output:\\
Solution: -52

\subsection{Algorithm}
\begin{easylist}
\ListProperties(Start1=1, Start2=1, Start3=1, Start4=1, Start5=1)
	& Take user input for order of the determinant and check its validity.
	& Declare a 2-D array (det[ ][ ]) of the given order.
	& Take user input for the elements using two nested loops.
	& Create an object of the DeterminantSolver class (d) and call the Solve(int[ ][ ] det) method, passing det as the parameter.
	& Store the value of Solve() in a variable and print it out as the solution.

\subsubsection*{int Solve(int[ ][ ] det):}
\ListProperties(Start1=1, Start2=1, Start3=1, Start4=1, Start5=1)
	& Declare an int 'rows' and set it equal to det.length
	& Check if rows = 1
		&& Return det[0][0] (Only element in the determinant)
	& Declare an int 'cols' and set it equal to rows.
	& Declare an int 'val' and set it equal to 0.
	& Run a loop from a = 0 to a  $<$ cols, incrementing a at each iteration.
		&& Declare a 2-D int array newDet[ ][ ] and set order equal to rows - 1.
		&& Run a loop from i = 1 to i $<$ rows, incrementing i at each iteration.
			&&& Run a loop from j = 0 to j $<$ cols, incrementing j at each iteration
				&&&& Check if  j $<$ a,
					&&&&& set newDet[i - 1][j] to det[i][j]	
				&&&& Check if j $>$ a
\ListProperties(Start5*=1)
					&&&&& set newDet[i - 1][j - 1] to det[i][j]
		&& set val to val + det[0][a]$\ast (-1^a) \ast$Solve(newDet)
	& Return val
\end{easylist}

\subsection{Code}

\begin{lstlisting}
import java.util.Scanner;

public class DeterminantSolver
{

	public static void main(String[] args)
	{
		Scanner s = new Scanner(System.in);
		
		System.out.println("Enter Order of Determinant");
		int n = s.nextInt();		//Takes input for order of the Determinant
		
		if(n <= 0)					//Checks validity of input
		{
			System.out.println("Invalid Size. Program will Exit");
			System.exit(0);
		}
		
		int[][] det = new int[n][n];	//Declares a determinant of given order
		
		System.out.println("Enter Elements, row-wise");	//Loop for entry of data into array
		for(int i = 0; i < n; i++)
			for(int j = 0; j < n; j++)
				det[i][j] = s.nextInt();
		
		DeterminantSolver d = new DeterminantSolver();	//Object of same class
		int sol = d.Solve(det);		//Passes determinant into recursive function Solve(int[][] det)
		
		System.out.println("\n\n Solution: " + sol);	
	}

	public int Solve(int[][] det)
	{
		int rows = det.length;		//Parses number of rows
		if(rows == 1)				//Returns sole element if order is 1
			return det[0][0];
		int cols = rows;
		
		int val = 0;	//To store value of row in consideration
		
		for(int a = 0; a < cols; a++)	//Runs a loop iterating through the columns
		{
			int[][] newDet = new int[rows - 1][cols - 1];	//Declares a new determinant of order one less than original passed into function
			
			for(int i = 1; i < rows; i++)	//Runs loop through the rows
			{		
				for(int j = 0; j < cols; j++)	//Runs loop through the columns
				{	
					if(j < a)					//Skips the row and column in consideration in first loop
						newDet[i - 1][j] = det[i][j];
					if(j > a)
						newDet[i - 1][j - 1] = det[i][j];
				}
			}
			val += det[0][a] * ((int) (Math.pow(-1, a)) * Solve(newDet));		//Adds the value of current element to val
		}
		return val;		//Returns value of determinant
	}
}
\end{lstlisting}


\end{document}