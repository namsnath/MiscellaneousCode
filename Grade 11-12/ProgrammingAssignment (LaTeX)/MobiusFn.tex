\documentclass[ProgramminAssignment.tex]{subfiles}
\begin{document}

\section{Mobius Function}
Mobius Function M(n) for a natural number n is defined as:
\begin{easylist}
\ListProperties(Start1=1)

	& M(n) = 1 if n = 1
	& M(n) = 0 if any prime factor of N is contained in n more than once
	& M(n) = (-1)p if n is a product of p distinct prime factors

\end{easylist}


Input:\\
78

Output:\\
-1

\[78 = 2*3*13\]
\[M(78) = (-1)^3 = -1\]

\subsection{Algorithm}
\begin{easylist}
\ListProperties(Start1=1, Start2=1, Start3=1, Start4=1, Start5=1)

	& Declare a Scanner object to take user input
	& Take user input and store it in an int (n)
	& Call primeFac(), passing n as an argument and print its value
	
\end{easylist}

\subsubsection*{void primeFac(int n):}
\begin{easylist}
\ListProperties(Start1=1, Start2=1, Start3=1, Start4=1, Start5=1)

	& Declare an int (primeCount) to store number of prime factors and initialize it to 0
	& If Number is Prime (using checkPrime()), return -1
	& Run a loop from i = 2, to n, incrementing i at each iteration
		&& if i is Prime
			&&& if n is divisible by i more than one time (two times here), return 0
			&&& if n is divisible by i one time, increment primeCount
	& return $(-1)^{primeCount}$
	
\end{easylist}	

\subsubsection*{boolean checkPrime(int n):}
\begin{easylist}
\ListProperties(Start1=1, Start2=1, Start3=1, Start4=1, Start5=1)

	& if n is 0 or 1, return false
	& if n is 2, return true
	& if n is a multiple of 2, return false
	& Run a loop from 3 to sqrt(n) and increment by 2 at each iteration
		&& If n is divisible by the current index, return false, else continue loop
	& return true
	
	
\end{easylist}	

\subsection{Code}
\begin{lstlisting}
import java.util.Scanner;
public class MobiusFn
{

	public static void main(String[] args)
	{
		Scanner s = new Scanner(System.in);	//To take user input
		MobiusFn f = new MobiusFn();	//Class Object
		
		System.out.println("Enter number to check:");
		int n = s.nextInt();	//Number to check
		s.close();
		
		System.out.println(f.primeFac(n));	//Prints value of M(n)
	}
	
	int primeFac(int n)	//Function to Factorize the number
	{
		int primeCount = 0;	//Number of Prime Factors
		
		if(checkPrime(n))	//Checks if the number is prime (has only 1 prime factor)
			return -1;
		
		for(int i = 2; i < n; i++)	//Loops through numbers till n to check for factors contained more than once
		{	
			if(checkPrime(i))	//Checks if the number is a prime
			{
				if(n % (i * i) == 0)	//factor present more than once
					return 0;
				if(n % i == 0)	//Factor present only once
					primeCount++;
			}	
		}
		
		return (int) Math.pow(-1, primeCount);	//returns (-1)^p
	}
	
	boolean checkPrime(int num)	//Function to check if number is prime
	{
		if(num == 1 || num == 0)
			return false;
		if(num == 2)
			return true;
		if(num % 2 == 0)
			return false;
		
		for(int i = 3; i * i < num; i += 2)
			if(num % i == 0)
				return false;
		
		return true;
	}

}

\end{lstlisting}
\end{document}