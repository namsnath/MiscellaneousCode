\documentclass[ProgramminAssignment.tex]{subfiles}
\begin{document}

\section{Rotate Matrix by 90 Degrees Clockwise}


Input:\\
\begin{tabular}{ccc}
1&	2&	3	\\
4&	5&	6	\\
7&	8&	9\\	
\end{tabular}

Output:\\
\begin{tabular}{ccc}
7&	4&	1\\	
8&	5&	2	\\
9&	6&	3	\\
\end{tabular}

\subsection{Algorithm}
\begin{easylist}
\ListProperties(Start1=1, Start2=1, Start3=1, Start4=1, Start5=1)

	& Take user input for order of matrix
	& Declare an array(A[ ][ ]) of appropriate size
	& Take user input for data
	& Print the original matrix
	& Run a loop from i = 0 to n - 1, incrementing i at each iteration
		&& Run a loop from j = n - 1 till j $\geq$ 0, decrementing j at each iteration
			&&& Print A[j][i]

\end{easylist}

\subsection{Code}
\begin{lstlisting}
import java.util.Scanner;
public class MatrixRotation90
{

	public static void main(String[] args)
	{

		Scanner s  = new Scanner(System.in);
		int m,n, i, j;
		//Dimensions
		System.out.println("Enter Order of Matrix:");
		n = s.nextInt();
		
		int A[][] = new int[n][n];	//Matrix
		
		System.out.println("Enter the elements in the array, Row-Wise:");
		//Data Entry
		for(i = 0; i < n; i++)
			for(j = 0; j < n; j++)
				A[i][j] = s.nextInt();
		
		System.out.println("Your Matrix:");
		//Original Matrix
		for(i = 0; i < n; i++)
		{
			for(j = 0; j < n; j++)
				System.out.print(A[i][j] + "\t");
			System.out.println();
		}
		
		System.out.println("Rotated Matrix:");
		for(i = 0; i < n; i++)
		{
			for(j = n - 1; j >= 0; j--)	//Transposes and prints rows in opposite direction
				System.out.print(A[j][i] + "\t");
			System.out.println();
		}
		
		s.close();
	}

}

\end{lstlisting}
\end{document}