\documentclass[ProgramminAssignment.tex]{subfiles}
\begin{document}

\section{Unique 2 Digit Combinations}
Write a program that asks the user to enter an integer and print all unique two digit 
integers that can be formed using the digits of that number

Input:\\
1231

Output:\\
11 12 13 21 23 31 32 

\subsection{Algorithm}
\begin{easylist}
\ListProperties(Start1=1, Start2=1, Start3=1, Start4=1, Start5=1)

	& Take user input of a number(num) and parse each digit to the elements of an array(digits[ ]) and store its length
	& Create a new StringBuffer(str)
	& Run a loop from i = 0 till length - 1, incrementing i at each iteration
		&& Append digits[0] and digits[i] with a blankspace to str
	& Run a loop from i = 1 to length - 2, incrementing i at each iteration
		&& Run a loop from j = i - 1 till j $\geq$ 0, decrementing j at each iteration
			&& Append digits[i] and digits[j] with a blank space to str
		&& Run a loop from j = i + 1 till j $\leq$ length, incrementing j at each iteration
			&& Append digits[i] and digits[j] with a blank space to str	
	& Run a loop from i = length - 2 till i $\geq$ 0, decrementing i at each iteration
	& Split the	StringBuffer into a String array from blank spaces
	& Parse the Strings into an int array(numbers[ ])
	& Sort numbers[ ] in ascending order
	& Create anew int array(uniques[ ]) to store unique elements
	& Compare elements in unique with numbers, adding all unique elements
	& Print unique elements

\end{easylist}

\subsection{Code}
\begin{lstlisting}
import java.util.Scanner;

public class Unique2DigitCombos
{

	public static void main(String[] args)
	{
		Scanner s = new Scanner(System.in);
		int i, j, temp;

		System.out.println("Enter a number");
		int num = s.nextInt();
		
		int length = String.valueOf(num).length();	//Length of input	
		
		int[] digits = new int[length];		//Array of digits
		for(i = 0; i < length; i++)	//Parses each digit to an elemnt in array
		{
			digits[i] = num % 10;
			num = num / 10;
		}
		
		StringBuffer str = new StringBuffer("");	//For editing
		
		for(i = 0; i < length; i++)
			str.append(digits[0] + "" + digits[i] + " ");	//Appends all digits with last one
		
		for(i = 1; i < length - 1; i++)	
		{
			for(j = i - 1; j >= 0; j--)	//iterates though digits and appends with all before them
				str.append(digits[i] + "" + digits[j] + " ");	
			for(j = i + 1; j <= length - 1; j++)	//iterates though digits and appends with all after them
				str.append(digits[i] + "" + digits[j] + " ");	
		}
		
		for(i = length - 2; i >= 0; i--)
			str.append(digits[length - 1] + "" + digits[i] + " ");	//Appends first digit with all before it
		
		String[] num_str = str.toString().split(" ");	//Splits StringBuffer into numbers
		int numbers[] = new int[num_str.length];	//Array for numbers
		
		
		for(i = 0; i < numbers.length;i++)
			numbers[i] = Integer.parseInt(num_str[i]);	//Parses each String to int
		
		for(i = 0; i < numbers.length; i++)		//Sorts in ascending order
		{
			for(j = 0; j < numbers.length - 1; j++)
			{
				if(numbers[j + 1] < numbers [j]) //Change this to "Array[j + 1] > Array [j]" to sort in descending order
				{
					temp = numbers[j + 1];
					numbers[j + 1] = numbers [j];
					numbers[j] = temp;
				}
			}
		}
		
	
		int[] uniques = new int[numbers.length];	//Array of unique numbers
		
		for(i = 0; i < numbers.length; i++)	//initializes all elements to 0
		{
			uniques[i] = 0;
		}
		uniques[0] = numbers[0];
		
		for(i = 1, j = 1; i < numbers.length && j < uniques.length - 1; i++)
		{
			if(numbers[i] == uniques[j])	//Compares consecutive number with element in uniques, skips if equal
				continue;
			else
			{
				j++;	//Go to next element
				uniques[j] = numbers[i];	//Set uniques to number
			}
		}
		
		for(i = 0; i < uniques.length; i++)
			if(uniques[i] == 0)
				continue;
			else
				System.out.print(uniques[i] + " ");
		
	}

}

\end{lstlisting}
\end{document}